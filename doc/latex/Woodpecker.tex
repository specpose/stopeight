\documentclass{report}
\usepackage{amsfonts}
\usepackage{amsmath}
\usepackage{amssymb}

\iffalse
\usepackage{biblatex}
\addbibresource{Stopeight.bib}
\fi
\usepackage{natbib}

\renewcommand{\baselinestretch}{1.25}

\begin{document}
\title{Woodpecker}
\author{Fassio Blatter}
\maketitle

\chapter{Single Graph Spectrogram}
The frequency content in an approximation can be evaluated by a spectrogram.\\\\
The spectrogram is calculated by applying a convolution transform to an artificially constructed signal for $t \in [a,b]$.
\begin{align}
rays(t)=ori(t)*\vert\iota_{X}(t)-\iota_{V}(t)\vert\\
woodpecker(\phi)=\cos(\arctan(\cot(\phi))) + \mathrm{i} arccot(\phi)\\
nautilus(a,b,\phi)=\int \limits _{-\infty}^{\infty}\int \limits _{a}^{b}rays(t)*woodpecker(\phi)\mathrm{d}t\mathrm{d}\phi
\end{align}
Unlike visibility, the orientation $ori(a,b)\neq ori(b,c)$ can change between two charts $U_{m},U_{m+1}$.


\iffalse
\printbibliography
\fi
\bibliography{Stopeight}{}
\bibliographystyle{plain}

\end{document}
