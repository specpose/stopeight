\documentclass{report}
\usepackage{amsfonts}
\usepackage{amsmath}
\usepackage{amssymb}

\iffalse
\usepackage{biblatex}
\addbibresource{Stopeight.bib}
\fi
\usepackage{natbib}

\renewcommand{\baselinestretch}{1.25}

\begin{document}
\title{Woodpecker}
\author{Fassio Blatter}
\maketitle

\chapter{Fourier Basis Alternative}
The $woodpecker$ can be used for switching bases in Fourier-like transforms. The angle $\phi$ has to be defined to the required scale, for example the diameter in Analyzer.\\\\
The $woodpecker$ is an artificially constructed (periodic signal) for $t \in [a,b]$.\\
Any base can be made periodic if the pseudo-periodic sections are expressed in local coordinates and either the angle or the radius is made smooth.\\
To keep the integration from maxing out, usually a complex basis with an absolute value of 1 is used.\\
Many (periodic) bases can be made to be within that range by inverting them, or for non-periodic bases a local transform with a unitary set to the supremum can be used.\\
The inversion also determines, whether the (Fourier-like) transform is inverted $operation^{-1}$ or forward.\\\\
Here it is used on the non-orthogonal transform $\vartheta$
\begin{align}
C_{\infty}=\inf_{Cliff \in X} \lvert \iota_{X}(t) \frac{\mathrm{d}}{\mathrm{d}t} \rvert; [a,b] \in Cliff\\
impossiblefigure(a,b)=\int \limits _{a}^{b} \lvert\vartheta \circ \iota_{X}(t)\rvert\mathrm{d}t\\
woodpecker^{-1}(\phi)=\cos(\arctan(\cot(\phi))) + \mathrm{i} arccot(\phi)\\
operation(\phi) \circ impossiblefigure(a,b)=impossiblefigure(a,b)*woodpecker
\end{align}


\iffalse
\printbibliography
\fi
\bibliography{Stopeight}{}
\bibliographystyle{plain}

\end{document}
