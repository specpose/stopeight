\documentclass{article}
\usepackage{amsfonts}
\renewcommand{\baselinestretch}{2}


\begin{document}
\title{antis}
\author{Fassio Blatter}

\maketitle


\renewcommand{\baselinestretch}{2}

Every zero crossing $x_{m}$ of the 2nd derivative

$\{x_{m}\}_{m \in \mathbb{Z} , f''(x_{m})=0}$

The Lebesgue for all the Intervals $x_{m}-x_{m-1}=T$

$\mu(\{x \vert f''(x_{m})=0, f''(x_{m-1})=0, x_{m}-x_{m-1}=T, x \le x_{m}, x \ge x_{m-1}\}) \mathrm{d} T$

For all the positive and negative function values $f(x)$ respectively

$\mu_{+}(\{{\mu \vert f(x) > 0}\}) \mathrm{d} T, \mu_{-}(\{{\mu \vert f(x) < 0}\})\mathrm{d} T$

Parcival's theorem

$\int\limits_{-\infty}^{\infty} \vert f(x) \vert ^2 \mathrm{d} x = 1/2\pi * \int\limits_{-\infty}^{\infty} \vert \hat{f}(1/T) \vert ^2  \mathrm{d} T$

Can be interpreted for signed function values as

$\int\limits_{-\infty}^{\infty} (\int\limits_{-\infty}^{\infty} \delta (t - T) \mathrm{d} t * (\int\int f(x) \mathrm{d} x \mathrm{d} \mu) )\mathrm{d} T = 1/2\pi * \int\limits_{-\infty}^{\infty} \hat{f}(1/T) \mathrm{d} T$

The convolution of the two complex fourier transforms is equal the convolution of two semi-pulse functions, and the negative part $\mu_{-}$ is interpreted as the imaginary part of the complex conjugate

$\int\limits_{-\infty}^{\infty}(\int\limits_{-\infty}^{\infty} \delta (t - T) \mathrm{d} t * (\int \int f(x) \mathrm{d} x \mathrm{d} \mu_{+} + \mathrm{i} \int \int f(x) \mathrm{d} x \mathrm{d} \mu_{-}))  \mathrm{d} T = 1/2\pi * \int\limits_{-\infty}^{\infty} ( \int\limits_{-\infty}^{\infty}  f(x) * (cos(x/T)+\mathrm{i} sin(x/T)) \mathrm{d} x ) \mathrm{d} T$

Minor $t$ is a unique period, not a point in time.

\renewcommand{\baselinestretch}{1}

This condition holds for every signal with frequency components separated in the time domain and symmetric in $mod(n/2)=0$ equal length parts. If this is not the case, we have either:

1. Longer wavelength pulses overlapping the frequency component, which results in a amplitude modulation of the combined signal.

2. Shorter wavelength pulses intersecting the segments of the frequency component, which results in frequency modulation of the combined signal.

3. Non-symmetric waveforms and/or different length half-pulses, which can not be removed with this procedure. See stopeight for further discussion of such signals.

(A longer wavelength, undetected harmonic can be excluded in this scenario since source separated harmonics are symmetric pulses of integer multiples of the wavelength and we imply that the shortest $symmetric$ pulse is found beforehand in the removal procedure below, i.e. the higher harmonics are $not symmetric$ because of criteria 1)

For efficient removal of individual pulses, the $amplitude$ of the pulse needs to be known. Longer wavelength pulses have the highest likelyhood to contain pulses which violate the above criteria, so their $amplitude$ can not be determined. We have to start removal with the shortest symmetric wavelength pulses.



\end{document}