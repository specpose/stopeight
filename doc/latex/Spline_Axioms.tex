\documentclass[a4paper,portrait]{report}
\usepackage{amsfonts}
\usepackage{amsmath}
\usepackage{amssymb}

\iffalse
\usepackage{biblatex}
\addbibresource{Stopeight.bib}
\fi
\usepackage{natbib}

\renewcommand{\baselinestretch}{1.25}
\begin{document}
\title{Spline Axioms}
\author{Fassio Blatter}
\maketitle

\chapter{Spline Approximations}
A set of vectors $\begin{pmatrix}x \\ y\end{pmatrix} \in X$ within any parametrisation bounds $t\in[a,b]$ can be translated, rotated and scaled by $M^{3x3}$, so that $\iota_{X}(a)=\begin{pmatrix}0 \\ 0\end{pmatrix}$ and $\iota_{X}(b)=\begin{pmatrix}1 \\ 0\end{pmatrix}$. An approximation (arc) $A$ of arbitrary polynomial degree is required.\\\\
For example, approximating a straight line has polynomial degree one.
\begin{equation}
line(t):
\begin{pmatrix}
1 \\
t
\end{pmatrix}
\underbrace{\begin{pmatrix}
1 & -1\\
0 & 1
\end{pmatrix}}_{Jordanian}
\begin{pmatrix}
M \begin{pmatrix} 0 \\ 0 \end{pmatrix} & M \begin{pmatrix} 1 \\ 0 \end{pmatrix}
\end{pmatrix}
=
\begin{pmatrix}
x \\
y
\end{pmatrix}
\end{equation}
Here, the matrix $M^{3x3}$ of affine transformations is the identity matrix. It will be omitted in the following.\\\\
The more complicated case of a quadratic bezier spline
\begin{equation}
quad(t):
\begin{pmatrix}
1 \\
t \\
t^2
\end{pmatrix}
\underbrace{\begin{bmatrix}
1 & -1 & 0\\
0 & 1 & -1\\
0 & 0 &1
\end{bmatrix}}_{Jordanian}
\underbrace{\begin{pmatrix}
x(S) & x(C_{1}) & x(E) \\
y(S) & y(C_{1}) & y(E) \\
0 & 0 & 0
\end{pmatrix}}_{Bezier Control Points}
=
\begin{pmatrix}
x \\
y \\
0
\end{pmatrix}
\end{equation}
The Jordanian is composed of the sum of identity matrices in upper right sub-spaces.
The characteristic polynomials of the diagonal and bidiagonal are composed of binomial coefficients.
\begin{align}
\chi_{\begin{pmatrix}1 & 0 & 0\\0 & 1 & 0\\0 & 0 & 1\end{pmatrix}} = \begin{pmatrix}3 \\ 0\end{pmatrix}*t^0 - \begin{pmatrix}3 \\ 1\end{pmatrix}*t^1 + \begin{pmatrix}3 \\ 2\end{pmatrix}*t^2 - \begin{pmatrix}3 \\ 3\end{pmatrix}*t^3\\
\chi_{\begin{pmatrix}-1 & 0\\0 & -1\end{pmatrix}} = \begin{pmatrix}2 \\ 0\end{pmatrix}*t^0 + \begin{pmatrix}2 \\ 1\end{pmatrix}*t^1 + \begin{pmatrix}2 \\ 2\end{pmatrix}*t^2
\end{align}
Note: It is a sparse approximation.\\\\
If start and end points are limited to be in the transversality $S,E \in Y$, the maximal polynomial degree can be reduced to four.
\begin{equation}
quart(t):
\begin{pmatrix}
1 \\ t \\ t^2 \\ t^3 \\ t^4
\end{pmatrix}
\underbrace{\begin{pmatrix}
1 & -1 & 0 & 0 & 0\\
0 & 1 & -1 & 0 & 0\\
0 & 0 &1 & -1 & 0\\
0 & 0 & 0 & 1 & -1\\
0 & 0 & 0 & 0 & 1
\end{pmatrix}}_{Jordanian}
H
=
\begin{pmatrix}
x \\ y \\ 0 \\ 0 \\ 0
\end{pmatrix}
\end{equation}
Note: Because we are only interested in planar coordinates, the tri- and quadridiagonals can be ignored. It is a polynomial of degree 4 in a 2 dimensional subspace (Sparse).\\\\
Axiom: Because of associativity, an inverse of the control points can be left multiplied. Because of associativity, the inverse of the Jordanian can be left multiplied.
\begin{equation}
quart(t):
\begin{pmatrix}
1 \\ t \\ t^2 \\ t^3 \\ t^4
\end{pmatrix}
=
\underbrace{\begin{bmatrix}
1 & 1 & 1 & 1 & 1\\
0 & 1 & 1 & 1 & 1\\
0 & 0 & 1 & 1 & 1\\
0 & 0 & 0 & 1 & 1\\
0 & 0 & 0 & 0 & 1
\end{bmatrix}}_{Jordanian^{-1}}
H^{-1}
\begin{pmatrix}
x \\ y \\ 0 \\ 0 \\ 0
\end{pmatrix}
\end{equation}

\chapter{Representation}

A quartic spline is fully defined by $S,E$ and three on spline control points $c_{1},c_{3} \in A;c_{2} \in C_{3}$. The corresponding Bezier control points $q_{1},q_{2} \in Q_{3}$ are obtained by substituting $t=1/4$,$t=2/4$ and $t=3/4$ with $c_{1}$,$c_{2}$ and $c_{3}$ in the quartic linear equation.
\begin{align}
H=\begin{bmatrix}
x(S) & x(q_{1}) & x(C_{3}) & x(q_{2}) & x(E) \\
y(S) & y(q_{1}) & y(C_{3}) & y(q_{2}) & x(E) \\
0 & 0 & 0 & 0 & 0\\
0 & 0 & 0 & 0 & 0\\
0 & 0 & 0 & 0 & 0
\end{bmatrix}\\
representation: S \times C_{3} \times E \rightarrow A \times V \times A\\
S,C_{3},E \in A; S,V,E \in \iota_{Y}\\
(S,C,E)\mapsto(c_{1},v_{1},c_{3})
\end{align}\\
The four quadratic splines $\gamma_{H_{1}} ... \gamma_{H_{4}}$ are defined by $S,E$ and on spline control points $c_{1},c_{2},c_{3},c_{5},c_{6},c_{7} \in A;c_{4} \in C_{3}$. The corresponding Bezier control points $q_{1},q_{2},q_{3},q_{4} \in Q_{1}$ for the pieces $H_{1},H_{2},H_{3},H_{4}$ are obtained by substituting $t=0,t=1/2$ and $t=1$ with $c_{1},c_{3},c_{5},c_{7}$ in each of the four quadratic linear equations.
\begin{align}
H_{1}=
\begin{bmatrix}
x(S) & x(q_{1}) & x(c_{2})\\
y(S) & y(q_{1}) & y(c_{2})\\
0 & 0 & 0
\end{bmatrix};
H_{2}=
\begin{bmatrix}
x(c_{2}) & x(q_{2}) & x(c_{4})\\
y(c_{2}) & y(q_{2}) & y(c_{4})\\
0 & 0 & 0
\end{bmatrix}\\
H_{3}=
\begin{bmatrix}
x(c_{4}) & x(q_{3}) & x(c_{6})\\
y(c_{4}) & y(q_{3}) & y(c_{6})\\
0 & 0 & 0
\end{bmatrix};
H_{4}=
\begin{bmatrix}
x(c_{6}) & x(q_{4}) & x(E)\\
y(c_{6}) & y(q_{4}) & y(E)\\
0 & 0 & 0
\end{bmatrix}
\end{align}

\iffalse
\printbibliography
\fi
\bibliography{Stopeight}{}
\bibliographystyle{plain}

\end{document}
