\documentclass{report}
\usepackage{amsfonts}
\usepackage{amsmath}
\usepackage{amssymb}

\iffalse
\usepackage{biblatex}
\addbibresource{Stopeight.bib}
\fi
\usepackage{natbib}

\renewcommand{\baselinestretch}{1.25}
\newcommand\norm[1]{\left\lVert#1\right\rVert}

\begin{document}
\title{Stopeight Comparator}
\author{Fassio Blatter}
\maketitle

\chapter{Introduction}
(Formula are work in progress. The present state of the software implementation includes working overlay comparison.)\\\\
This file is in the Stopeight repository on Github. Please edit here:\\
https://github.com/specpose/stopeight/tree/master/doc/latex/Stopeight\_Comparator.tex\\
The DOI can be found here ~\cite{Analyzer}.\\\\
Initially the aim of this paper was to make a dataset from each of two $separated$ signal sources comparable. The sources stem from two mixed signals, becoming comparable using geometric overlay comparison (currently implemented). Following the simple time-based example in the Sphinx docs, this happens regardless of the scale of correlated time and amplitude stretch, but preserving frequency modulation and amplitude ratios (More in theory of Grapher).\\
It has now become evident that the methods in this paper can provide the $separation$ task itself to some extent. The content of this paper now generally applies to datasets in the Cartesian plane. Specifically it refers to addition operations in the Complex plane, such as the Fourier integrands in Stopeight Grapher\cite{Grapher}[1.1] .\\
In the course of specifying the Stopeight Analyzer transversality, which serves as foundation for the subsequent procedures, some more aspects have taken shape. These aspects are encapsulated in the definition Isolation and Visibility. They are an indication of the fractal dimension ~\cite{Widon} and cutoff of the approximation. On the other hand the methods described in Antis \cite{Antis} may provide an alternative separation indicator not relying on the Stopeight Analyzer transversality.\\
Finally, the role of Compact Covers has become part of a speculation, that signals ultimately need bounds of a maximal polynomial degree for comparison and that the Turns $T_{4}$ and Corners $C_{4}$ could be viable for neural networks. The reason for this assumption is founded on the theory that the Stopeight Analyzer transversality exhibits a Causal Structure.

\chapter{Multiresolution}
Visibility and Isolation can be calculated on multiple resolutions between all possible bounds $a,b$.
Visibility and Isolation are based on chained functions.
\begin{align*}
function() \circ Analyzer(sn) \circ vectorgraph(bs)\\
(bs)=m/(npartitions);(sn)=(nsides)
\end{align*}
Note: $h \circ g(a) \circ f(b)$ is the functional notation for $h(g(a,f(b)))$.\\
Note: $m,npartitions$ are defined in Stopeight Grapher~\cite{Grapher}[1.2] and $Max_{Straight}$ depending on $nsides$ are defined in Stopeight Analyzer~\cite{Analyzer}[3.23].\\\\
The resolutions are based on summing block size $m/npartitions$ (Visibility) and straightness calibration $nsides$ for the unit circle (Isolation). They disappear at the maximum resolution, which is the approximation where nothing is straight (Isolation) and the Vector Graph where the summing block size is 1 (Visibility).\\\\
Visibility and Isolation are covariant. To address this issue we keep straightness calibration $sn$ in Visibility and blocksize $bs$ in Isolation respectively fixed.
\begin{align}
visibility_{sn} \circ Analyzer(sn) \circ vectorgraph(bs)\\
isolation_{bs} \circ Analyzer(sn) \circ vectorgraph(bs)
\end{align}
Note: $h_{a} \circ g(a) \circ f(b)$ is the functional notation for $h(g(a=const,f(b)))$.

\section{Fake Isolation}
For every resolution, a Vector Graph with adjusted blocksize $bs$ has to be calculated and Stopeight Analyzer has to be applied.
Isolation is the path integral on the approximation.
\begin{align}
fakeisolation(a,b)= dir_{Y} \int \limits _{a}^{b} \lvert \xi(t)\frac{\mathrm{d}}{\mathrm{dt}} \rvert \mathrm{d}t
\end{align}
For a Spiral this would be a viable solution.
For a ZigZag, the Edges would start shifting.
For a Swell, the Crests, together with the Edges induce massive shifting.
This approach is not viable.

\section{Fake Visibility}
For every resolution, a Vector Graph has to be calculated and the Stopeight Analyzer straightness $sn$ has to be calibrated. $V$ is a point on the base-line $S,E$ defined by the focal axis in $C$.
Visibility is the area function under the approximation.
\begin{align}
fakevisibility(a,b)= ori \int \limits _{a}^{b} \vert \xi(t)-V(U_{m})\vert  \mathrm{d}t
\end{align}
Note: This implicit definition uses differential geometry. For a discrete implementation, use something like the Algorithm Version for curvature, but with center $V(U_{m})$ instead of $S(U_{m})$.\\\\
This is true if the outer bounds $S_{0},E_{n}$ of the Compact Cover are enforced. $t_{0}$ and $t_{m}$ have to be aligned with these bounds.
This approach is also not viable.

\section{Total Derivative}
The total derivative here is relatively simple, unlike ~\cite{Analyzer}[3.1] the parameters $sn,bs$ don't depend on $t$.
\begin{align}
\frac{\mathrm{d}}{\mathrm{d} t} function(t,sn,bs)\\
= \frac{\mathrm{d}}{\mathrm{d} t} function(t) \circ Analyzer(sn) \circ vectorgraph(bs)\\
= \frac{\partial}{\partial t} function(t) \circ ( \frac{\partial}{\partial sn}  \frac{\mathrm{d} }{\mathrm{d}t} * Analyzer(sn) \circ (\frac{\partial}{\partial bs} \frac{\mathrm{d}}{\mathrm{d}t}*vectorgraph(bs)) )
\end{align}
Note: The derivative is continuous (kompakter traeger), but not smooth (stetig). Therefore the antiderivative does not exist.\\\\
But the outcome of $Analyzer$ still depends on $vectorgraph$, so for every interval $[a,b]$, there are two variables $\partial sn$ and $\partial bs$ for each function $fake visibility, fake isolation$, so this approach is not viable for a combined 2D spectrogram!
\begin{align}
\frac{\partial}{\partial bs\partial sn} fakevisibility(a,b),\frac{\partial}{\partial sn\partial bs} fakeisolation(a,b)
\end{align}

\section{Hausdorff Dimension}

From a geometric point of view, Visibility is the area under the curve. Intuitively it also appears that Isolation is simply the formula of Straightness, but applied on the approximation $\xi$ instead of the original Vector Graph $\iota_{X}$.\\\\
Think of Isolation as a Koch Curve. The Hausdorff Dimension of two path integrals would be the ratio of two path integrals with different blocksize $bs$. However this may become difficult to compare as bounds are inevitably shifting with two non overlying approximations. Remember that Isolation is $directly$ depending on blocksize $bs$. Blocksize adjustment $removes$ artefact.\\
The area function under the curve is the Nautilus area. Instead of a ratio, we take the difference of two approximations on their own respective chart centers $V\circ U_{m}(Analyzer)$. This performs even worse under shifting bounds as we calibrate straightness $sn$. Remember that Visibility is $directly$ depending on straightness $sn$. Straightness calibration $omits$ artefact at a certain level.\\\\
Consider: A vectorgraph resembling the solution for an ordinary differential equation of one variable. The vectorgraph has a start because it is a Anfangswertproblem and a blocksize $bs$ corresponding to the differential $\mathrm{d}$. The straightness calibration $sn$ would simplify the solution.

\chapter{Coordinate Transforms}
The coordinate transforms break the global affine space $(AffineSpace,(X,\mathbb{R}^2,\oplus,\odot),\overrightarrow{\text{ }})$ into charts $U_{m}$. For the orthogonal coordinate transform $\upsilon$ the transversality remains the same. For the non-orthogonal coordinate transform $\vartheta$, a slightly adjusted transversality is created.

\subsubsection*{Simulation}
In order to get from function graph analysis to the Stopeigh Analyzer, geometric analysis was required. In order to understand Visibility and Isolation, the Stopeight Analyzer is not enough, and some of the phenomena in Simulation have to be looked at beforehand.

A $backside dune$ has a hill-shape figure on the frontside and a straight section on the backside. The dune has a Visibility which does not change in a short time-frame. The wind is blowing sand over the hill-top and Isolation breaks up on the straight part. There is a cause and effect relationship between $neighboring$ segments.

A $backside crest$ in the formation of ocean swell creates gravitational energy on the backside of multiple wave-packets which in turn modifies the surface tension on the frontside of the same wave-packets. There is a cause and effect relationship between $neighboring$ sections, but it affects all the wave packets in the wave-train.\\
Example: In an electronic circuit, this would have to be implemented using a feedback convolution.

A $carriage wheel$ can get hitched if it bumps against a tree-root. Prediciting where this increased Visibility will reduce the Isolation of the wheel can not be ultimately concluded, just by looking at the wheel. There is only a probability that the wheel will start to break, when it encounters a backward thrust against its wooden spines. It is therefore not at all a neighboring relationship of cause and effect.

\subsection{Orthogonal}
The point of view transformation $V(U_{m}')$ to $\overrightarrow{SE}/2$ is a rotation to the x-axis and a Translation to the center. The adjustment of the complex radius $r=\sqrt{x^2+y^2}$ is performed by the Translation.
\begin{align}
\underbrace{
\begin{pmatrix}
0 & 0 & 0 & 0 & -x(V-S) \\
0 & 0 & 0 & 0 & -y(V-S) \\
0 & 0 & 0 & 0 & 0 \\
0 & 0 & 0 & 0 & 0 \\
0 & 0 & 0 & 0 & 0
\end{pmatrix}
}_{Translation}
*(
\underbrace{
(M^{5x5} * H^{5x5})^{-1}
}_{\text{global to local for spline}}
*\begin{bmatrix} x \\ y \\ 0 \\ 0 \\ 0 \end{bmatrix})
\end{align}
Note: The rank of M and H is lower than their dimensions, but their product has full rank, so it is invertible.\\\\
The rotation angle $\phi$ is applied afterwards using the translated coordinates. Instead of doing a Translation to $S$, followed by a Rotation by $\phi(S',E')$ and another Translation by $-V'$, we are doing a single Translation to $V$, followed by a Rotation by $\phi(V',E')$.
\begin{align}
\phi_{x}(P_{1},P_{2}) = -ori\lvert \mathrm{arccos}(x(P_{2})/\lvert \overrightarrow{P_{1}P_{2}}\rvert)\rvert\\
\upsilon
\begin{pmatrix}x \\ y \\ 0 \\ 0 \\ 0\end{pmatrix}
=
\underbrace{
\begin{pmatrix}
\mathrm{cos}(\phi_{x}(V',E')) & -\mathrm{sin}(\phi_{x}(V',E')) & 0 & 0 & 0 \\
\mathrm{sin}(\phi_{x}(V',E')) & \mathrm{cos}(\phi_{x}(V',E')) & 0 & 0 & 0 \\
0 & 0 & 0 & 0 & 0 \\
0 & 0 & 0 & 0 & 0 \\
0 & 0 & 0 & 0 & 0
\end{pmatrix}
}_{Rotation}*
Translation((MH)^{-1}\begin{bmatrix} x \\ y \\ 0 \\ 0 \\ 0 \end{bmatrix})
\end{align}
Note: The coordinate transform with an orthogonal, but individual and local basis is followed by identifying $\mathbb{R}^2=\mathbb{C}$. This is a polar to cartesian transform $r\circ (Translation\circ \iota_{T}(t_{S},t_{E})),\phi \circ (Rotation\circ (\iota_{T}(t_{S},t_{E}),Translation(r)))$.\\\\
To make the complex angle $\phi$ smooth connecting the charts $U_{m}\cap U_{m-1}$ we take the angle of the last vector $E\in U_{m-1}$.
\begin{align}
\phi (U_{m}) = \phi (E \in U_{m-1})+\phi(U_{m})\\
\phi (U_{0})=\phi(U_{0}) -\pi
\end{align}
Notation: Here, $\phi (U_{m})$ is short for $\{\upsilon (z) \}_{z \in U_{m}}$
\subsection{Non-Orthogonal}
If the global coordinates with the basis vectors $(1,0)$ and $(0,1)$ are transformed into the local coordinates of each chart with the basis vectors $\overrightarrow{VE(U_{m}')}/\lvert \overrightarrow{VE(U_{m}')}\rvert$ and $\overrightarrow{VC(U_{m}')}/\lvert \overrightarrow{VC(U_{m}')}\rvert$, the complex sum of a positive $ori(\xi)=1$ $backsidecrest$ and its mirror negative $ori(\xi)=-1$ $backsidecrest$ zero out in the imaginary dimension.\\
Likewise two consecutive half-circles facing opposite orientations $ori$ zero out in the real part within a single chart and zero out in the imaginary part over the two consecutive charts.\\
To make the complex radius $r$ smooth (stetig), the new bases for both coordinates $x,y$ are made unitary by dividing them by the base lengths.
\begin{align}
\phi_{y}(P_{1},P_{2}) = ori(\frac{\pi}{2}-\lvert \mathrm{arccos}(x(P_{2})/\lvert \overrightarrow{P_{1}P_{2}}\rvert)\rvert)\\
\vartheta
\begin{pmatrix}x/\overline{V'E'} \\ ori(y/\overline{V'C'}) \\ 0 \\ 0 \\ 0\end{pmatrix}=
\underbrace{
\begin{pmatrix}
\mathrm{cos}(\phi_{x}(V',E')) & -\mathrm{sin}(\phi_{x}(V',E')) & 0 & 0 & 0 \\
\mathrm{sin}(\phi_{y}(V',C')) & \mathrm{cos}(\phi_{y}(V',C')) & 0 & 0 & 0 \\
0 & 0 & 0 & 0 & 0 \\
0 & 0 & 0 & 0 & 0 \\
0 & 0 & 0 & 0 & 0
\end{pmatrix}
}_{Rotation}*
Translation((MH)^{-1}\begin{bmatrix} x \\ y \\ 0 \\ 0 \\ 0 \end{bmatrix})
\end{align}
Note: The coordinate transform to a non-orthogonal basis is the equivalent of a cartesian to polar transform $y\circ (Rotation\circ (Translation(r),\xi(t_{C}))),x\circ (Rotation\circ (Translation(r),\iota_{T}(t_{S},t_{E})))$.\\
Note: The function $r(t)$ is continuous (auf Kompaktem Traeger) $and$ smooth (stetig).

\section{Impossible Figures}

\subsection*{Compact Cover}

If you would happen to try to run the Stopeight Analyzer recursively on a ZigZag, you would find out that you get a valid result from the output as well, because the output from the Stopeight Analyzer still contains original points from $\iota_{X}$ (except for inserted Corners in Straights). But acknowledge that theoretically you would have treated Turns $T$ as Corners $C$ (see Spiral). The result is that you are realigning focal axes $\overline{VC}$. Essentially you would have increased the polynomial degree of the approximation, but this is not desirable in ZigZags. As an alternative you could try the Woodpecker Transform \cite{Woodpecker} on a ZigZag (needs computational verification).\\\\
If you would happen to look for a signal expansion on a coordinate transform in a Spiral, you would probably try to look for a common center of all the Cliffs. You would then have a function $\phi (r)$, which would have an analytic meaning, for example on a signal base $e^{ik};k \in \mathbb{Z}$.\\\\
Even though Visibility and Isolation are present in the two examples above ($Fake Visibility=0$ for ZigZag and $Fake Visibility=\infty$ for a repeating perfect circle), they $directly$ depend on each other only in Swells.

\subsection{Breaking The Transversality}

With the coordinate transforms, it is now possible to not only integrate and multiply $\xi$, but also $\iota_{X}$. The benefit of this is that you can get back data. Data that has been previously removed by $\xi$. Applying this to anything that is too far off from what can be approximated by a quartic polynomial spline of a single variable, should be avoided. While it is desirable to increase the polynomial degree of an approximation, this should not come at the loss of a causal structure.\\\\
Consider: Buoy data is not inherently well suited for quartic approximations. The Stopeight Analyzer is useless. But with large $r$ representing the infimum of the path integral of a $C_{\infty}$ approximation, the infimum corresponds to Corners. All the data lying between local $T_{\infty}$ is already smooth (stetig) and could be integrated using the non-orthogonal coordinate transform $\vartheta$, thus giving complex visibility and isolation.\\
Note: It is not advisable to switch to a norm of the complex angle $\norm{\phi} = \phi_{m}-\phi_{m-1}$ in this buoy example, because complex isolation would disappear. $sn$ would be more accurately represented by linear regression and $bs$ would create angular distortion.\\\\
Impossible Figures break the transversality $Y$. The original Euclidean norm $\norm{\cdot}_2 = \lvert z_{m+1}-z_{m} \rvert$ is still the same. But the taxicab norm $\norm{\cdot}_1 = \sum \lvert z_{m} \rvert$ works on bounds $T \not \in Y,Z; T \in X$.

\section{Complex Isolation and Visibility}
There are two main modes of integration. The cartesian mode is also useful for simulation. Forward and backward $dir_{Y}$ signals can be added together. Also consider multiplying TCT points of a forward $dir_{Y}=1$ and backward $dir_{Y}=-1$ approximation in Stopeight Simulation.
\begin{align}
cartesian(a,b)=\int \limits _{a}^{b} \vartheta (x(t)+\mathrm{i}y(t)) \mathrm{d} t\\
(((\int,\norm{z}=\norm{z}_2),\oplus),\norm{t}=\norm{z}_1)
\end{align}
Notation: There is absolutely no difference in infinitesimal $\int \limits _{a}^{b} \mathrm{d}i$ notation and discrete $\sum  \limits _{i=a}^{b}$ notation if a metric space with a distance norm $(\mathbb{R},\norm{i}=d(a,b))$ is defined. A set $\mathbb{R}$ with bounds $a,b$ has to be open $a,b \not \in \mathbb{R}$, closed $a,b \in \mathbb{R}$ or clopen $a \in \mathbb{R}, b \not \in \mathbb{R} \lor a \not \in \mathbb{R},b \in \mathbb{R}$.\\\\
The cartesian mode should now show $Complex Isolation$ as the intensity of the single, combined 2D spectrogram and $Complex Visibility$ as the vertical shift. However, its derivative is still not smooth (stetig). A spectrogram can be achieved, but it does not lead to an algebra based on its antiderivative.
\begin{align}
\frac{\partial}{\partial bs\partial sn} cartesian(a,b)\label{eq:3}
\end{align}
\subsection{Plancherel Identity}
An algebraic (computational) proof, whether the Plancherel identity applies to a set of complex numbers comprises the following conditions.\\\\
If $z_{1}+z_{2}\in (x+\mathrm{i}0)$ and $\lvert z_{1}+z_{2}\rvert=diameter$ and $\lvert z_{1}*z_{2}\rvert=(diameter)^2$, $z_{1}$ and $z_{2}$ could be corresponding Fourier bounds for a polar norm $\norm{\cdot}=\phi$.
The Plancherel identity only applies to Fourier Series. Fouries Series imply that the sums of the components of the signal expansion be zero on the respective component's periodic bounds. It makes a link between the complex radius (signal energy) and the complex angle (convolved frequency components).
\subsection{Signal Decomposition}
A viable signal decomposition of the $vectorgraphs$ is now possible based on $visibility$ and signal graph inverse (See below). Moreover the signal decomposition is retaining the bounds of the underlying $vectorgraph$. However it is still not a full blown signal expansion. The bin-size doesn't matter at practical resolutions. It could be desirable to use different colors for Spirals, Swells and ZigZags.

\section{Phase Retaining Signal Expansion}
The polar mode is
\begin{align}
polar(a,b)=\int \limits _{-\infty}^{\infty} \lvert \upsilon (x(t)+\mathrm{i}y(t))\rvert \mathrm{e}^{\mathrm{i}\phi} \mathrm{d}\phi\\
(((\int,\norm{z}=\norm{r}_1),\odot),\norm{\phi}= \norm{t}_2)\Rightarrow \norm{r}_1 \sim \norm{\phi}
\end{align}
Note: This is not a double integral! A double integral would be $\int \int \mathrm{d}y\mathrm{d}x$ with $(\mathbb{R}^2,\norm{x},\norm{y})$.

\subsection{Parseval Theorem}
A geometric proof.\\\\
Consider the $z_{n}$ of a $waveform$ \cite{Grapher}[3.1] representing two consecutive, perfect half circles facing opposite orientation $ori$ on the $xaxis$, where $t_{1}=0$ is the start of the first half-circle and $t_{2}=2\pi$ is the end of the second half-circle. This is the maximum signal $r(t)$ to be expected from the Analyzer.
\begin{align}
z_{n} = \int \limits _{a}^{b}(\cos \frac{2\pi}{(t_{2}-t_{1})}t)+ \sin \frac{2\pi}{(t_{2}-t_{1})}t) \mathrm{d}t;a,b\geq t_{1} \land a,b\leq (t_{2}-t_{1})/2 \land a\leq b\label{eq:5}\\
z_{n} = \int \limits _{a}^{b}-(\cos \frac{2\pi}{(t_{2}-t_{1})}t)+ \sin \frac{2\pi}{(t_{2}-t_{1})}t) \mathrm{d}t;a,b\geq (t_{2}-t_{1})/2 \land a,b\leq t_{2} \land a\leq b
\end{align}
Note: The sequence $\{z_{n}\}$ is entirely real.\\\\
Factor out the complex radius by dividing with the Fourier base of the corresponding frequency.
\begin{align}
z_{n} = \int \limits _{t_{1}}^{t_{2}} \underbrace{\lvert \upsilon \circ waveform_{1}(a,b)\rvert}_{\text{here }\pi /2} \mathrm{e}^{\mathrm{i}2\pi t/(t_{2}-t_{1})}\mathrm{d}t\label{eq:4}
\end{align}
Note: The $phaseprofile_{t_{2}-t_{1}}$ (from start to end) is entirely real.\\\\
$phaseprofile_{t_{2}-t_{1}}$ ($m=n\Rightarrow \text{1 vector}$) is the inverse Fourier transform $\mathrm{e}^{\mathrm{i}\omega t}$
\begin{align}
z_{0} = \hat{f}^{-1}(1/\tau)\circ \underbrace{\lvert \upsilon \circ waveform_{1}(a,b)\rvert}_{\text{orthogonal }\upsilon} = \int \limits _{t_{1}}^{t_{2}} \sin \frac{2\pi t}{\tau} + \mathrm{i} \sin^2 \frac{2\pi t}{\tau} \mathrm{d}t
\end{align}
Notation: $\tau$ is the period of the frequency $\omega=\frac{1}{\tau}$, not a Turn $T$ from the Analyzer.\\
Note: Choosing a $waveform$ $bs$ other than 1 is not required, because there is no jitter (perfect).\\
Note: The period $\tau$ is only relevant to the Fourier basis $e^{\mathrm{i}t}$, not the waveform. Since the $phaseprofile$ $is$ the Fourier basis, setting $bs$ of $phaseprofile$ is the equivalent of adjusting $\tau$ and setting $sn$ of $waveform$ is a low pass filter.\\\\
Reinterpreting the representation of the approximation from the $Analyzer(sn=1) \circ waveform(bs=1)$ \eqref{eq:4} as a waveform, the Parseval theorem states
\begin{align}
\int\limits_{-\infty}^{\infty} \lvert \underbrace{\hat{f}^{-1}(1/\tau)\circ \upsilon(\xi (t))}_{\text{what we've just done}}\rvert ^2 \mathrm{d}\tau = 1/2\pi * \int\limits_{-\infty}^{\infty} \lvert \underbrace{\upsilon (\xi (t))}_{\text{what we started with}} \rvert ^2 \mathrm{d}t\label{eq:6}\\
(((\int \int,\norm{z}=\norm{r}_2),\odot),\norm{\phi}= \sup \phi)
\end{align}
Note: $waveform$ $bs=1$ (perfect) and $sn=1$ (nothing is Straight).\\\\
Reinterpreting the waveform \eqref{eq:5} as a signal $\xi (t) - \iota_{T}(t)$, we assume (geometric proof) that the (Riemann) integral of the signal is proportional to the integral of the complex radius of the orthogonal transform.
\begin{align}
\int\limits_{t_{1}}^{t_{2}} \lvert \underbrace{y(\xi '(t))}_{\text{rotated to }xaxis}\rvert \mathrm{d} t \sim \int\limits_{t_{1}}^{t_{2}} \lvert \underbrace{\upsilon (\xi (t))}_{\text{circle from center}} \rvert \mathrm{d}t\label{eq:7}
\end{align}
Reinterpreting the signal $y(\xi '(t))$ \eqref{eq:7} as an arbitrary signal $s(t);a \leq t \leq b$ and substituting into \eqref{eq:6}, there exists a $phaseprofile_{\tau}$ for every frequency in an $arbitrary$ signal and a $combined$ orthogonal transform.
\begin{align}
\int\limits_{-\infty}^{\infty} \lvert \underbrace{\hat{f}^{-1}(1/\tau)\circ \hat{s}(t)}_{\text{individual phaseprofile}}\rvert ^2 \mathrm{d}\tau \sim 1/2\pi * \int\limits_{-\infty}^{\infty} \lvert \underbrace{\upsilon \circ vectorgraph_{1}(a,b)}_{\text{combined transform}} \rvert ^2 \mathrm{d}t
\end{align}
Note: Setting $bs=\tau$ is possible here, because $\mathrm{d}t$ integration bounds formally do not get exhausted (open $(a=-\infty,b=\infty)$).\\\\
If we start from a signal, we can construct a $combined$ vectorgraph by multiplying with the Fourier basis
\begin{align}
\underbrace{ \hat{f}_{\tau}\circ s}_{\hat{s}(t)}(a,b): (x,y)_{m/(b-a)} = \frac{1}{2\pi} \int\limits_{a}^{b} \lvert s(t)\rvert \mathrm{e}^{-\mathrm{i}2\pi t/\tau} \mathrm{d}t\\
combined_{bs=(b-a)}(t_{1},t_{2}): (x,y)_{m/bs} = \int \limits_{-\infty}^{\infty} \hat{s}_{\tau}(a,b) \mathrm{d}\tau\\
derivative(t_{1},t_{2}): (x,y)_{(m-1)/bs} = \int \limits_{-\infty}^{\infty} \hat{s}_{\tau}(a,b) \frac{\mathrm{d}}{\mathrm{d}(bs)} \mathrm{d}\tau
\end{align}

\subsection{Discussion}
$phaseprofile$ $bs$ removes artifact (Fourier transform window function), $phaseprofile$ $sn$ omits artifact (covariant low pass filter). Both are operating on shifting bounds, which does not require exact bounds like Fourier Series, but \eqref{eq:3} is not smooth (stetig). (is smooth $\Rightarrow \exists$ antiderivative?; invertible $\Rightarrow \exists$ reconstruction)\\\\
Instead of the signal expansion from the outer Fourier transform integral, the Stopeight Analyzer is doing a polynomial signal expansion. It is not doing it on a smooth generic $\phi$ base, but getting smooth $r(t)$ from normalisation in the non-orthogonal coordinate transform.\\\\
Note: A Fourier transform is an attempt to make the continuous, but non-smooth function $s(t)$ integrable, but it does $not$ provide an antiderivative?\\\\
Note: Only when $s(t)$ is smooth (and continuous), there can be a corresponding Fourier Series for that signal.

\section{Neural Networks}
The presence of a signal base is a requirement for neural networks. Both the points in $Y\cup Z$ and the coordinate transforms provide bases. The oriented manifold $dir_{Y}$ of the approximation is a limitation for use in neural networks.\\\\
Note: A norm is a constraint on a space affecting a prospective differential. A base is an affine or non-affine transform mapping the prospective integrand. A norm for a frequency is a polar norm. A base for signal energy has a cartesian norm and is composed of the sum of local bases.\\\\
The key to achieving a good neural network inference rate lies in finding a consistent autocalibration $bs,sn$ algorithm for both inference and training data. Because the antiderivative in \eqref{eq:3} does not exist, a scalogram with $xaxis=bs$ and $yaxis=sn$ has to be calculated for each individual signal $s(t)$.\\\\
Note: $sn$ is fully defined (continuous) in $[1,\infty)$. $bs$ can theoretically be made continuous by interpolation, but it is only defined in $[4,m]$ and has the irregularities when $(m\mod n) > 0$.

\chapter{Signal}
\section{Graph Inverse}
The full resolution $vectorgraph_{1}$ is considered. Start of the signal $s(t);t=0$ and end of the signal $s(t);t=m$ are needed.
\begin{equation}
(vectorgraph_{1})^{-1}: \mathbb{R}^2 \rightarrow \mathbb{R}
\end{equation}
The signal reconstruction is lossless, albeit normalized.
\section{Approximation Inverse}
Start of the signal, end of the signal and approximation $\xi(t)$ are needed for interpolation. There is a loss of information in Arcs $A$ caused by Straights and Spikes.
\begin{equation}
vectorgraph^{-1}: \mathbb{R}^2 \rightarrow \mathbb{R}
\end{equation}
It is possible to reconstruct a signal from the approximation, effectively making it a suitable format for compression or even upsampling. It can also make Vector Graphs, which have not been obtained from a signal, audible or suitable for 1-dimensional analysis.\\
Note: The compression is dynamic. In some cases, the data may actually increase.

\chapter{After Separation}
Asymetric, non-repetitive half-pulses, which can be compared by matchline overlay comparison.\\\\
Sisyphus: A Foray into Fractals\\
Let's take the example of Sisyphus rolling a rock up a parabolic hill. We measure the altitude of the rock as a function of time. Sisyphus will roll the rock up the hill much slower than it will roll down the hill. Therefore variance of the Vector Graph in one direction is much lower than going in the other direction, but the periodicity of the falling depends on the rising.\\
If we compare this to the formation of mountain ridges and their erosion, we can still say they depend on each other, but the periodicity is much more disconnected. The formation of mountain ridges depends on the flow of magma and ultimately on the thermomagnetic energy within the planet. Erosion, however depends on a much shorter period, which depends on the availability of atmosphere and ocean and the electromagnetic energy directed at them by the sun.\\
The Vector Graph is non-symmetric in regard of time. Because phase varies there is no frequency; We speak of periodicity. An acoustic impulse propagates symmetrically as it oscillates between two fixed values. We can therefore assume that it has a fixed variance in both directions. This criteria is not true for our geology example. We may have to use different means to find correlations. This is where the Stopeight Comparator comes in handy.\\

\iffalse
\printbibliography
\fi
\bibliography{Stopeight}{}
\bibliographystyle{plain}

\end{document}
