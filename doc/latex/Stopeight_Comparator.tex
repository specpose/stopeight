\documentclass{report}
\usepackage{amsfonts}
\usepackage{amsmath}
\usepackage{amssymb}
\usepackage{hyperref}

\iffalse
\usepackage{biblatex}
\addbibresource{Stopeight.bib}
\fi
\usepackage{natbib}

\renewcommand{\baselinestretch}{1.25}
\newcommand\norm[1]{\left\lVert#1\right\rVert}

\begin{document}
\title{Stopeight Comparator}
\author{Fassio Blatter}
\maketitle

\chapter{Introduction}
(Formula are work in progress. The present state of the software implementation includes working overlay comparison.)\\
(Todo: Introduce formal definition of measure spaces)\\\\
This file is in the Stopeight repository on Github. Please edit here:\\
\href{https://github.com/specpose/stopeight/tree/master/doc/latex/Stopeight\_Comparator.tex}{https://github.com/specpose/stopeight/tree/master/doc/latex/Stopeight\_Comparator.tex}\\
The DOI can be found here ~\cite{Stopeight}.\\\\
Initially the aim of this paper was to make separated sources in a mixed signal comparable using either numeric or geometric overlay comparison (currently implemented), regardless of time and amplitude stretch, but preserving frequency modulation and amplitude ratios.\\
In the course of specifying the Stopeight Analyzer transversality, which serves as a foundation for the subsequent procedures, some more aspects have taken shape. These aspects are encapsulated in the definition of two complex numbers, isolation and visibility. Visibility may lead to a method of approximating polynomials of arbitrary degree. Isolation is an indication of the fractal dimension and cutoff of the approximation.

\chapter{Isolation}
Isolation was meant to be the ratio of the arc-lengths of the approximation and the input $\xi'(t)/\iota'_{X}(t)$.\\\\
The square root of the product $\sqrt{\xi'(t)*\iota'_{X}(t)}$ happens to be the geometric mean.
\begin{equation}
isolation(a,b)=\int \limits _{a}^{b} \sqrt{\xi'(t)*\iota'_{X}(t)} \mathrm{d}t
\end{equation}
Note: Taking the mean instead of the ratio has a few implications. Lets assume we have computed the Stopeight Analyzer transversality by applying the affine transformations in a positive direction $dir_{Y}=1$. We now measure this against the actual input in the same direction $dir_{X}=1$. The resulting number is entirely real.\\\\
The antiderivative exists. It is smooth (~\cite[Riemann Integrable]{Widon}). Therefore a convolution can be applied \emph{after} integration.\\
It can have values between $[0,\inf]$, which is useful for this convolution:\\
\begin{equation}
woodpecker(a,b)=isolation(a,b)
\end{equation}
Note: If we had chosen a product of geometric means instead of the convolution would make that isolation grows if the means are $(0,1]$, or even worse canceled out for zeros. A sum would not amplify the effect of large geometric means in short intervals.

\chapter{Visibility}
Visibility is computed by creating an auxiliary set of chart-wise points
\begin{equation}
h: U_{m} \mapsto [a,b] \mapsto O_{m}
\end{equation}
Note: Because of Corner insertion (\cite{Stopeight}[3.20]), $card(U_{m}) \geq 3$.\\\\
They are Centres of Cliffs by the arithmetic mean,\\
\begin{align}
O_{m}=\{(x',y') \in \mathbb{R}^2 \vert \min \norm{(x,y)-(x',y')}_{2};(x,y) \in U_{m} \}\\
\norm{\cdot}_{2}=\sqrt{(x-x')^2 + (y-y')^2}
\end{align}
\begin{align}
visibility(a,b)= \sum_{i=0}^{m} \frac{\int \limits _{a}^{b}\norm{\xi(t)-O_{m}}_{2}* (\cos(angle(t)) + \mathrm{i}\sin(angle(t))) \mathrm{d}t}{\int \limits _{a}^{b}\norm{\iota_{X}(t)-O_{m}}_{2}* (\cos(angle(t)) + \mathrm{i}\sin(angle(t))) \mathrm{d}t}\\
angle(t)=\arctan\frac{x(\gamma(t)-O_{m})}{y(\gamma(t)-O_{m})}
\end{align}
Note: Integration \emph{after} convolution. Division of two complex numbers is partial/total derivative?

\chapter{Polynomials and Splines}
Finding their Handrolls (chart on Spiral, tangential space).
\begin{align}
C_{O}=\sup_{t \in U} \lvert \frac{\mathrm{d}^2}{\mathrm{d}t^2}visibility(t) \rvert \approx 0
\end{align}
Finding Eliptic Joints.\\
\begin{align}
T_{O}=\inf_{t \in V} \lvert \frac{\mathrm{d}^2}{\mathrm{d}t^2}visibility(t) \rvert \approx 0
\end{align}

\subsection{Recursion}
($alt(a,b)$ products across interval spaces?)\\\\
Typically, we do $signs$ over scale-spaces by setting the domain to $dom(\iota,\xi)=K_{p}$.
Which is reflected in the $\mu(\gamma_{dom})$ of sum of a function over a scale space.\\
\begin{equation}
signs(\mu_{+},function )= \sum \limits _{\inf \limits _{\mu_{+}} (b-a)}^{\sup \limits _{\mu_{+}} (b-a)} function (a,b)
\end{equation}
$times$ is the product of a function over a interval space $dom(\iota,\xi)=T_{n}$.
\begin{equation}
times(\mu_{+},function) = \prod_{\inf \limits _{\mu_{+}} (b-a)}^{\sup \limits _{\mu_{+}} (b-a)} function(a,b)
\end{equation}

\subsection{Imaginary Visibility}
Visibility is an increase of the polynomial Coefficients(?) of a curve segment.
(No: When visibility is increased in a Spiral, it creates more subsegments/scale spaces). Points in $Y$ become uncertain at around $2\pi/3$. It is a Cantor-like diminishing isolation.
\begin{align}
visibility(a,b)= alt(a,b) * signs(\mu,solidangle_{Rel}(a,b))\\
visibility(a,b)= alt(a,b) * signs(\mu,curvature(a,b)-Q(a,b))(?)
\end{align}
(Lebesgue on curvature? $Max_{Curve}$ based or sign based?)
\subsection*{Cases}
When curvature is decreased in a Spiral, it becomes a ZigZag.
\subsection*{Scale Space}
$area$ grows, but $diameter$ constant.

\subsection{Real Isolation}
(No: A lack of isolation is directly linked to the creation of more interval spaces.)
\begin{align}
isolation(a,b) = ori(a,b)*times(\mu,\frac{1}{area_{jitter}(a,b)})\\
isolation(a,b) = ori(a,b)*times(\mu,straightness(a,b))(?)
\end{align}
(Frame bundle orthonormal?)
\subsection*{Interval Space}
$jitter$ grows, but $diameter$ constant.

\subsection{Jaggedness}
Is aliasing of discretisation/rasterisation, but not to be confused with signal aliasing. Can be defined on Focals and Turns.
\begin{align}
jaggedness(a,b)=\int \limits _{a}^{b} \lvert \iota_{X}(t)-\xi_{F}(t)\rvert \mathrm{d}t
\end{align}

\section{Indicator}
In order for Stopeight overlay comparison to work, source separation has to be performed. A complex function measuring separation can be defined as.
\begin{equation}
separation(a,b) = isolation(a,b) + \mathrm{i} visibility(a,b)
\end{equation}
Important: This can be calculated while resampling. (See Grapher)\\\\
Eventually, bounds of periodicity could be found. The directed waves of straightness and curvature going forward/backward synchronously are periodic.
\begin{align}
periodicity(a,b) =  (\int \limits _{a}^{b} \xi(t)-\iota_{X}(t) \mathrm{dt}) *(visibility(a,b) +\mathrm{i} isolation(a,b))
\end{align}
Or directly on the signal?
\begin{align}
periodicity(a,b) =  (\int \limits _{a}^{b} s(t) \mathrm{dt}) *(visibility(a,b) +\mathrm{i} isolation(a,b))
\end{align}
Note: $visibility$ and $isolation$ already include the trigonometric functions of the polar to cartesian transform.\\\\
Consider: A polar to cartesian transform is a convolution with a sum of two fundamental numeric Taylor Series Expansions. Sine and Cosine functions are intrinsically linked to the quadratic entity $\sqrt{x^2 + y^2} =1$ and dividing the circle by four (90 deg.). Hyperbolic Sine is linked to $1/x$.\\\\
Explanation:\\
The imaginary part is the other side of the wave vs\\
The imaginary part is the phase shift (fft)\\\\
(multiplication? addition?)
\begin{align}
(a,b) \mapsto znet
\end{align}

\subsection*{Complex Norm}
A complex norm $\norm{\cdot}$.
\begin{align}
\norm{\cdot} : \mathbb{C} \rightarrow \mathbb{R}\\
\end{align}
Banach Space
\begin{align}
\norm{z} = \sqrt{Re(z)^2+Im(z)^2}
\end{align}
Hilbert Space
\begin{align}
\norm{z} = z*ComplexConjugate(z)=1
\end{align}

\subsection*{After Separation}
Asymetric, non-repetitive pulses, which can be compared by matchline overlay comparison (See Sisyphos)\\\\
Sisyphus: A Foray into Fractals\\
Let's take the example of Sisyphus rolling a rock up a parabolic hill. We measure the altitude of the rock as a function of time. Sisyphus will roll the rock up the hill much slower than it will roll down the hill. Therefore variance of the Vector Graph in one direction is much lower than going in the other direction, but the periodicity of the falling depends on the rising.\\
If we compare this to the formation of mountain ridges and their erosion, we can still say they depend on each other, but the periodicity is much more disconnected. The formation of mountain ridges depends on the flow of magma and ultimately on the thermomagnetic energy within the planet. Erosion, however depends on a much shorter period, which depends on the availability of atmosphere and ocean and the electromagnetic energy directed at them by the sun.\\
The Vector Graph is non-symmetric in regard of time. Because phase varies there is no frequency; We speak of periodicity. An acoustic impulse propagates symmetrically as it oscillates between two fixed values. We can therefore assume that it has a fixed variance in both directions. This criteria is not true for our geological example. We may have to use different means to find correlations. This is where the Stopeight Comparator comes in handy.\\


\iffalse
\printbibliography
\fi
\bibliography{Stopeight}{}
\bibliographystyle{plain}

\end{document}