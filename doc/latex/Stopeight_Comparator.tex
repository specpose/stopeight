\documentclass{report}
\usepackage{amsfonts}
\usepackage{amsmath}
\usepackage{amssymb}

\iffalse
\usepackage{biblatex}
\addbibresource{Stopeight.bib}
\fi
\usepackage{natbib}

\renewcommand{\baselinestretch}{1.25}

\begin{document}
\title{Stopeight Comparator}
\author{Fassio Blatter}
\maketitle

\chapter{Introduction}
(Formula are work in progress. The present state of the software implementation includes working overlay comparison.)\\\\
This file is in the Stopeight repository on Github. Please edit here:\\
https://github.com/specpose/stopeight/tree/master/doc/latex/Stopeight\_Comparator.tex\\
The DOI can be found here ~\cite{Analyzer}.\\\\
Initially the aim of this paper was to make a dataset from each of two $separated$ signal sources comparable. The sources stem from two mixed signals, becoming comparable using geometric overlay comparison (currently implemented). Following the simple time-based example in the Sphinx docs, this happens regardless of the scale of correlated time and amplitude stretch, but preserving frequency modulation and amplitude ratios (More in theory of Grapher).\\
It has now become evident that the methods in this paper can provide the $separation$ task itself to some extent. The content of this paper now generally applies to datasets in the Cartesian plane. Specifically it refers to addition operations in the Complex plane, such as the Fourier integrands in Stopeight Grapher\cite{Grapher}[1.1] .\\
In the course of specifying the Stopeight Analyzer transversality, which serves as foundation for the subsequent procedures, some more aspects have taken shape. These aspects are encapsulated in the definition of two multiresolution spectrograms, Isolation and Visibility. They are an indication of the fractal dimension ~\cite{Widon} and cutoff of the approximation. On the other hand the methods described in Antis \cite{Antis} may provide an alternative separation indicator not relying on the Stopeight Analyzer transversality.\\
Finally, the role of Compact Covers has become part of a speculation, that signals ultimately need bounds of a maximal polynomial degree for comparison and that the Turns $T_{4}$ and Corners $C_{4}$ could be viable for neural networks. The reason for this assumption is founded on the theory that the Stopeight Analyzer transversality exhibits a Causal Structure.

\chapter{Multiresolution Spectrogram}
Visibility and Isolation can be calculated on multiple resolutions between all possible bounds $a,b$. The resolutions are based on summing block size (Visibility) and straightness calibration (Isolation). They disappear at the maximum resolution, which is the Aproximation where nothing is straight (Isolation) and the Vector Graph where the summing block size is 1 (Visibility). Visibility and Isolation are covariant. To address this issue we keep straightness calibration $polygon n=\infty$ in Visibility and summing block size $j=1$ in Isolation respectively fixed.

\section{Isolation}
For every resolution, a Vector Graph has to be calculated and the transversality applied.
Isolation is the integral of the absolute difference of the input and approximation.
\begin{equation}
isolation_{m/2^j}(a,b)=\int \limits _{a}^{b} \vert\iota_{X}(t)-\xi(t)\vert \mathrm{d}t
\end{equation}
Note: The antiderivative exists. It is smooth (~\cite[Riemann Integrable]{Widon}).\\\\

\section{Visibility}
For every resolution, the transversality for $vectorgraph_{1}$ has to be recalibrated. When $Max_{Straight}$ falls below the area of the smallest triangle of consecutive points in $X$, the maximum resolution is reached.
$V$ is the intersection of the focal axis and the base-line $S,E$. Visibility is the integral of the absolute difference of the focal intersection and the approximation.
\begin{align}
visibility_{m/2^j}(a,b)= \int \limits _{a}^{b} \vert \xi(t)-\iota_{V}(t)\vert  \mathrm{d}t
\end{align}
The integrand does not change in the case of a geometric circle, but it does in the case of a geometric spiral.\\
For a Spiral type Compact Cover, the transversality is irregular.
\begin{align*}
\iota_{V}(t) =
\begin{cases}
(S,V_{0}) & t \in [t(S),t(C_{4})]\\
(V_{0},V_{1}, ... , V_{n}) & t \in [t(C_{4}),t(C_{4})]\\
(V_{n},E) & t \in [t(C_{4}),t(E)]
\end{cases}
\end{align*}
Note: It would also make sense to intersect two adjacent focal axes directly $V'$.\\\\
For a Swell or a ZigZag type Compact Cover, the transversality is as expected.
\begin{align*}
\iota_{V}(t) =
\begin{cases}
(S,V_{0},E) & t \in [t(S),t(E)]\\
(S,V_{1},E) & t \in [t(S),t(E)]\\
...\\
(S,V_{n},E) & t \in [t(S),t(E)]\\
\end{cases}
\end{align*}
Note: The outer bounds $S_{0},E_{n}$ of the Compact Cover are enforced. $t_{0}$ and $t_{m}$ have to be aligned with these bounds.

\chapter{Vector Graph Inverses}
\section{Graph Inverse}
The full resolution $vectorgraph_{1}$ is considered. Start of the signal $s(t);t=0$ and end of the signal $s(t);t=m$ are needed.
\begin{equation}
(vectorgraph_{1})^{-1}: \mathbb{R}^2 \rightarrow \mathbb{R}
\end{equation}
The signal reconstruction is lossless, albeit normalized.
\section{Approximation Inverse}
Start of the signal, end of the signal and approximation $\xi(t)$ are needed for interpolation. There is a loss of information in Arcs $A$ caused by Straights and Spikes.
\begin{equation}
vectorgraph^{-1}: \mathbb{R}^2 \rightarrow \mathbb{R}
\end{equation}
It is possible to reconstruct a signal from the approximation, effectively making it a suitable format for compression or even upsampling. It can also make Vector Graphs, which have not been obtained from a signal, audible or suitable for 1-dimensional analysis.\\
Note: The compression is dynamic. In some cases, the data may actually increase.

\chapter{Factorisation}
Axiom: Complex multiplication preserves signal characteristics even in time-variant scenario. Rescaling is just a single factor, but depending on the sampling rate, the frames are affected differently.
\begin{align}
\prod \limits _{U \in \xi'} C_{4}-V;C_{4}-V \in \mathbb{C}^1\\
\prod \limits _{U \in \xi} E-S;E-S \in \mathbb{C}^1
\end{align}
What pulling together or subdividing factors is doing to an audio signal $s(t)$ is not yet evaluated. There could be a possibility of quick search in Vector Graph factor databases.
\section{Binomials}
Binomial factor decomposition works for complex numbers. Particularly formula with changing signs are of interest because of their application to the multiplication of complex conjugates $(x+\mathrm{i}y)(x-\mathrm{i}y)=x^2+y^2$, $(x+\mathrm{i}y)(x-\mathrm{i}y)(x+\mathrm{i}y)$, etc.

\chapter{After Separation}
Asymetric, non-repetitive half-pulses, which can be compared by matchline overlay comparison.\\\\
Sisyphus: A Foray into Fractals\\
Let's take the example of Sisyphus rolling a rock up a parabolic hill. We measure the altitude of the rock as a function of time. Sisyphus will roll the rock up the hill much slower than it will roll down the hill. Therefore variance of the Vector Graph in one direction is much lower than going in the other direction, but the periodicity of the falling depends on the rising.\\
If we compare this to the formation of mountain ridges and their erosion, we can still say they depend on each other, but the periodicity is much more disconnected. The formation of mountain ridges depends on the flow of magma and ultimately on the thermomagnetic energy within the planet. Erosion, however depends on a much shorter period, which depends on the availability of atmosphere and ocean and the electromagnetic energy directed at them by the sun.\\
The Vector Graph is non-symmetric in regard of time. Because phase varies there is no frequency; We speak of periodicity. An acoustic impulse propagates symmetrically as it oscillates between two fixed values. We can therefore assume that it has a fixed variance in both directions. This criteria is not true for our geological example. We may have to use different means to find correlations. This is where the Stopeight Comparator comes in handy.\\


\iffalse
\printbibliography
\fi
\bibliography{Stopeight}{}
\bibliographystyle{plain}

\end{document}
