\documentclass{report}
\usepackage{amsfonts}
\usepackage{amsmath}
\usepackage{amssymb}

\iffalse
\usepackage{biblatex}
\addbibresource{Stopeight.bib}
\fi
\usepackage{natbib}

\renewcommand{\baselinestretch}{1.25}
\newcommand\norm[1]{\left\lVert#1\right\rVert}

\begin{document}
\title{Stopeight Grapher}
\author{Fassio Blatter}
\maketitle

\chapter{Definitions}
\section{Vector Graph}
Consider a function that produces a sequence of independent vectors with coordinates $x,y \in \mathbb{R}$.
\begin{equation}
Anything \rightarrow \{(x,y)_{m}\}_{m \in \mathbb{N}}
\end{equation}
Note: It is a vector space with the Euclidean norm $((X,\mathbb{R}^2,\oplus,\odot),\norm{\cdot}_2)$. Setting a norm $\norm{\cdot}_2$ implies that the distance $\sqrt{x^2+y^2}$ can be $divided$ into $n \in \mathbb{R}$ vector parts with a zero remainder $m \mod n = 0$. This type of condition on a norm extends $\odot$ so that a unitary base $\lvert \overrightarrow{base} \rvert$ can be generated for all vectors $(x,y)$ of a metric space $(\mathbb{R}^2,\oplus,\odot)$.\\\\
The Vector Graph of the sequence of vectors is its appended form. The vectors are translated by the preceding ones.\\
\begin{equation}
vectorgraph_{1}: (x,y)_{m=n}=\sum_{i=0}^{n-1} (x,y)_{i}
\end{equation}
Note: The vector space preserves the taxicab norm $((X,\mathbb{R}^2,\oplus,\odot),\norm{\cdot}_2,\norm{\cdot}_1)$.
Note: Setting a norm $\norm{\cdot}_1$ implies the total number of elements $m$ can be $divided$ into $n \in \mathbb{Z}$ pieces with a zero remainder $m \mod n = 0$. However the base from the vector space $\lvert \overrightarrow{base} \rvert$ does not have a corresponding vector $\overrightarrow{base}$ in the metric space $(R^2,\norm{\cdot}_{1})$. It can be used, like in the procedings for the total traversal distance of a vectorgraph.
\section{Partitioning}
The appended form can, but does not have to be, subdivided into $n$ partitions of blocksize $m/n$ with $m\mod n=0$. The vectors in the partition are sumed and the resulting vector is stretched by the effective length of the taxicab norm of the vectors in the partition.
\begin{align}
partitioning: ((X,\mathbb{R}^2,\oplus,\odot),\norm{\cdot}_2,\norm{\cdot}_1) \rightarrow (((X,\mathbb{R}^2,\oplus,\odot),\norm{\cdot}_2),\norm{\cdot}_1)\\
partitioning: vectorgraph_{1} \mapsto vectorgraph_{m/n}\\
vectorgraph_{m/n}:(x,y)_{n}=\sum_{i=0}^{n-1} (\underbrace{\norm{(x,y)_{(i+1)*m/n}}_{1} - \norm{(x,y)_{i*m/n}}_{1}}_{length} * \underbrace{\frac{\sum_{k=i*m/n}^{(i+1)*m/n} (x,y)_{k}}{\vert \sum_{k=i*m/n}^{(i+1)*m/n} (x,y)_{k} \vert}}_{direction})
\end{align}
Algorithm Version: Affine 2D scaling transformation (to be disclosed)\\\\
This ensures that the traversal distances are preserved!
\begin{equation}
(vectorgraph_{m/n_{1}},\norm{\cdot}_{1})=(vectorgraph_{m/n_{2}},\norm{\cdot}_{1})\label{eq:9}
\end{equation}
\chapter{Vector-length Variant}
A variant version is composed of a sequence of $p$ Vector Graph pieces on the same taxicab norm $(\{X_{p}\}_{p\in \mathbb{Z}},\norm{\cdot}_1)$. Not only the traversal distances of the individual pieces \eqref{eq:9}, but also the traversal distance of the whole graph has to be kept constant.
\begin{align}
(m/n)_{0}\frac{\norm{X_{0}}_{1}}{(m/n)_{0}}+...+(m/n)_{p}\frac{\norm{X_{p}}_{1}}{(m/n)_{p}}=\norm{variantgraph}_1
\end{align}
Notation: $\norm{X_{p}}$ is short for $\norm{\max \limits _{X_{p}} x}$
\section{Equalising Length}
It may be desirable to work with vectors of the same length. This may be required for inverting the Vector Graph $vectorgraph^{-1}$, comparing two different Vector Graphs or using a Hilbert (dot product) space algebra.\\\\
To avoid interpolation, integer multiples of the vector lengths over all pieces would have to be used.
\begin{equation}
\sqrt{(x^2+y^2)} \mod \inf \limits _{X} \sqrt{(x^2+y^2)} = 0
\end{equation}
Clearly, choosing a blocksize $m/n$ for a piece $q$ that results in the occurrence of a block that is shorter than the highest Euclidean distance in any particular piece $p$ of the entire sequence $\{X_{p}\}$ is not recommended for comparison because of the missing data in that particular piece.
\begin{equation}
(m/n)_{q}*\inf \limits _{X} \sqrt{(x^2+y^2)} \geq \sup \limits _{X_{p}} \sqrt{(x^2+y^2)}\label{eq:4}
\end{equation}
\section{Approximations}
Variance can be dealt with, because the ~\cite[Stopeight\_Analyzer.tex]{Analyzer} seamlessly ignores missing values.
\subsection{Piecewise Calibration}
The ~\cite[Stopeight\_Analyzer.tex]{Analyzer} not only ignores missing values that are not provided by the $vectorgraph$, but can also omit values, which it considers Straight. The goal of fitting multiple pieces together is to make the entire sequence comparable ~\cite[Non-Orthogonal]{Comparator}. The proportion of the blocksize and the straightness adjustment of each piece $p$ of the sequence is an important consideration when fitting individual pieces together.
\begin{equation}
\frac{(m/n)_{p}}{\norm{X_{p}}_{1}} \sim \frac{(nsides)_{p}}{\norm{X_{p}}_{1}}
\end{equation}
Note: $Max_{Straight}$ depending on $nsides$ is defined in Stopeight Analyzer~\cite{Analyzer}[3.23].
\subsection{Comparator Constraint}
The calibration $nsides$ leads to a major change of arc-length and the traversal distance has to be adjusted. The choice of representation of the approximation also leads to a minor change in arc-length, similar to the (Riemann) integration remainder $C$.
\begin{align}
(\norm{Analyzer_{nsides}(X_{0})}_{1}+C_{0})+...+(\norm{Analyzer_{nsides}(X_{p})}_{1}+C_{p})=\norm{vectorgraph_{(m/n)}}_1
\end{align}
Running the Analyzer on the individual pieces $X_{p}$ and dividing the approximation into $n$ equal blocks of size $m/n$ not only eliminates variance, but also sets a norm on the arc-length which is extensively used as a $constraint$ in Stopeight Comparator~\cite{Analyzer}.

\chapter{Usage Scenarios}
Notation: The rest of the proceedings in this paper includes the preliminary stage of creating a Vector Graph $(x,y)_{n}$ from a signal (function of one variable) $s(t)$.
\section{FT Domain (Time) Scalability}
This $relative$ Vector Graph has to be constructed from a signal $s(t)\in \mathbb{R}$ by appending vectors to each other.
\begin{align}
waveform_{m/n}: (x,y)_{n}=\sum \limits _{i=0}^{n-1}(Re(z_{i}),Im(z_{i}))
\end{align}
The complex numbers $z_{n}$ are obtained from a Polar to Cartesian transform. The angles are scaled to the maximum expected angle in the specific vectorgraph.
\begin{align}
z_{n}=\sum \limits _{i=n*m/n}^{(n+1)*m/n}r_{i}\cos(\phi_{i})+r_{i}\mathrm{i}\sin(\phi_{i})\\
\phi_{n}=\frac{s(t_{n})-s(t_{n-1})}{\sup \lvert s(t_{m})-s(t_{m-1}) \rvert}*\pi+\phi_{n-1};\phi_{0}=0\\
r_{n}=t_{n}-t_{n-1};r_{0}=0
\end{align}
Note: The convolution is a Riesz space. It multiplies a factor on both sides $(X,\oplus,\odot,\leq)$.
\subsection{Finding Bounds}
In a one dimensional mixed signal $s(t)$, there are quiet, low frequencies. They are difficult to spot because of the presence of high frequency noise. Creating a Vector Graph provides a high precision solution to the adequate time-localisation of these pulses, and even half pulses where the wavelength is not known.
\subsection{Pitch}
The benefit of this method is that a sequence of variable wavelength pulses can be compared at different pitches. The unbound independent scalar is exposing a wavelength/amplitude correlation (unlike just amplitude; See \eqref{eq:3}).
\subsection{Compression}
Image Variance (to be disclosed).
\section{FT Image (Amplitude) Scalability}
This $absolute$ Vector Graph has to be constructed by appending vectors to each other.
\begin{align}
phaseprofile_{m/n}: (x,y)_{n}=\sum \limits _{i=0}^{n-1}(\mathrm{Re(z_{i})},\mathrm{Im(z_{i})})
\end{align}
The complex numbers $z_{n}$ are obtained from inverse Fourier Transform of $n$ partitions (Polar to Cartesian). The frequency multiplier is set to the frame size.
\begin{align}
z_{n}= \sum \limits _{i=n*(m/n)}^{(n+1)*(m/n)} s(t_{i})*(\cos(\frac{2\pi}{m/n}t_{i})+\mathrm{i}\sin(\frac{2\pi}{m/n}t_{i}))\label{eq:3}
\end{align}
Note: There is a leaking effect if the block size $m/n$ is not an integer multiple of the period of the signal $s(t)$ ~\cite[Fensterfunktion]{Fensterfunktion}. To some extent this can be solved by adjusting the partition size $m/n$ according to visibility and isolation in ~\cite[Stopeight\_Comparator.tex]{Comparator}\\\\
Note: The convolution is a Riesz space. It multiplies a factor on both sides $(X,\oplus,\odot,\leq)$.
\subsection{Harmonics}
The benefit of this method is that an audio feature in a mixed signal $s(t)$ can test positive in geometric overlay comparison regardless of the amplitude of the feature.

\iffalse
\printbibliography
\fi
\bibliography{Stopeight}{}
\bibliographystyle{plain}

\end{document}
