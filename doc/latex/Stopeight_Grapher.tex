\documentclass{report}
\usepackage{amsfonts}
\usepackage{amsmath}
\usepackage{amssymb}
\usepackage{hyperref}

\iffalse
\usepackage{biblatex}
\addbibresource{Stopeight.bib}
\fi
\usepackage{natbib}

\renewcommand{\baselinestretch}{1.25}
\newcommand\norm[1]{\left\lVert#1\right\rVert}

\begin{document}
\title{Stopeight Grapher}
\author{Fassio Blatter}
\maketitle

\chapter{Vector Graph}
Consider a function that produces a sequence of independent vectors with coordinates $x,y \in \mathbb{R}$.
\begin{equation}
Anything \rightarrow \{(x,y)_{m}\}_{m \in \mathbb{Z}}
\end{equation}
Note: It is a metric space with the Euclidean norm $(\mathbb{R}^2,\norm{\cdot}_2)$.\\\\
The Vector Graph of the sequence of vectors is its appended form. The vectors are translated by the preceding ones.\\
\begin{equation}
vectorgraph_{\log_{2}(m)}: (x,y)_{n}=\sum_{i=0}^{n} (x,y)_{i}
\end{equation}
Note: It is a metric space with a taxicab norm $(X,\norm{\cdot}_1)$.\\\\
The appended form can, but does not have to be, subdivided into $2^j$ partitions of size $m/2^j$. The vectors in the partition are sumed and the resulting vector is stretched by the effective length of the taxicab norm of the vectors in the partition.
\begin{equation}
vectorgraph_{j}:(x,y)_{n}=\sum_{i=0}^{n} (\norm{(x,y)_{(i+1)*m/2^j}}_{1} - \norm{(x,y)_{i*m/2^j}}_{1} * \frac{\sum_{k=i*m/2^j}^{(i+1)*m/2^j} (x,y)_{k}}{\vert \sum_{k=i*m/2^j}^{(i+1)*m/2^j} (x,y)_{k} \vert})
\end{equation}
This ensures that the traversal distances are preserved!
\begin{equation}
(vectorgraph_{j},\norm{\cdot}_{1})=(vectorgraph_{j+1},\norm{\cdot}_{1})
\end{equation}
Even a distance-variant version of the Vector Graph can be thought of. It is composed of a sequence of $p$ Vector Graph pieces on the same taxicab norm $(\{X_{p}\}_{p\in \mathbb{Z}},\norm{\cdot}_1)$.\\\\
To avoid interpolation, integer multiples of the Euclidean norm $\norm{\cdot}_{2}$ over all sequences would have to be used.\\\\
Clearly, choosing a $j$ that results in a Euclidean distance smaller than the highest Euclidean distance in the entire sequence $\{X_{p}\}$ is not recommended for comparison because of missing data.
\begin{equation}
(m/2^j)*\inf \limits _{X} \sqrt{(x^2+y^2)} \geq \sup \limits _{\{X_{p}\}} \sqrt{(x^2+y^2)}
\end{equation}

\chapter{Implementations}
The function $vectorgraph(a,b)$ is a angle function with relative angles scaled by
\begin{align}
\pi*(\sup_{s(t)} s'(t)-\inf_{s(t)} s'(t))
\end{align}

\iffalse
\printbibliography
\fi
\bibliography{Stopeight}{}
\bibliographystyle{plain}

\end{document}