\documentclass{report}
\usepackage{amsfonts}
\usepackage{amsmath}
\usepackage{amssymb}

\iffalse
\usepackage{biblatex}
\addbibresource{Stopeight.bib}
\fi
\usepackage{natbib}

\newcommand\norm[1]{\left\lVert#1\right\rVert}
\renewcommand{\baselinestretch}{1.25}
\begin{document}
\title{Simulation}
\author{Fassio Blatter}
\maketitle

\chapter{Introduction}
The interaction of physical processes can often be attributed to the interaction of polynomials of higher degree. The representation and its inverse ~\cite[Spline\_Axioms.tex]{Axioms} provides reflective areas. The theory here is that these areas $interact$ with each other in proximity. In a steady state, they can be approximated by polynomials. If movement is introduced, they become $dynamical systems$. For the reflective areas have implications with attractors and differential equations. I will provide three examples, together with some very general physics descriptions and at last, cases that have to be discussed.

\chapter{Examples}
\section{Wave}
Analysing the path of a Swell, composed of two adjacent Crests which would have been almost Cliffs, if not caught by Crest detection. The two adjacent splines have opposite orientations regarding $\xi'$. The second Vector Graph is a static boulder resting on the shore. In open water, the moving Swell is completely invisible with the positive half facing forward. The boulder is a single Swell, composed of a single Cliff. Once these two Vector Graphs meet each other, a change in the $moving$ Vector Graph is induced. As the negatively oriented half of the moving Swell encounters the anti-reflective area of the $static$ Swell, visibility is added to the positive half. Since the two Cliff-like Crests already have maximal curvature for the pulse-length, isolation is broken at the joints. Thus, a plethora of new frequencies is inserted into the moving Vector Graph. It is important to note that a specific wave-length may interact with just one peculiar boulder shape.
\section{Storm}
Weather systems have been simulated successfully using the Lorentz-Attractor. The surrounding splines of weather systems are quite distant. Vortices of polynomials of a high degree can exist, creating the so-called eye of the hurricane, which surprisingly is an area of calm. This is not uncommon in fluid dynamics. In fact the most stable entities, therefore most reliable areas of forecast are these $interaction$ areas.
\section{Atom}
A frequency modulated signal $s(t)$ traveling from an output, through a single atom, to a measurement device is measured. If the atom does not absorb or refract any of the spectrum, the signal should match exactly. If a single pulse is introduced matching the absorption spectrum, the event could be located in time and studied. This would be a single event as opposed to a periodic frequency and eventually the object being studied would vanish.

\chapter{Conversion}
The mathematically most intriguing oriented submanifolds are Crests and Spikes. We introduce a mechanism for the annihilation and creation of pulses. Physical phenomena such as the creation of enthalpy and enthropy during phase transitions could be modeled. A Crest would appear in the outgoing phase when a signal of enthropy brakes in the corresponding medium of the carrier. A Spike is a beat that appears in joint states of crystal grids when the incoming phase accumulates in a $different$ medium.
Traveling wave trains are moving over stationary reef chains. Upon creation, enthropic wave trains are impulses, at the least even a part of a half-pulse (Example Spiral). Enthalpic reef chains change state when they are converted to a plasma state for an infinitesimal short period of time.\\\\
A wave train appears in a wave packet that splits up. A wave packet that splits, decreases its isolation and transfers a part of its visibility to the sub-wave packets. Depending on the wavelength of a Reef, the original wave packet may never appear again.\\\\
On the other hand, the sub-wave packets of an open ocean wave pick up wind. The enclosing wave packet can form a new wave train, together with neighbouring wave packets and so slowly, the little left-over, chaotic visibility of the sub-wave packets gets transferred not only to the enclosing wave packet, but to the whole wave train. The swell gets groomed and the short wavelengths disappear.\\\\
Because in any accurate measurement of matter, only a limited number of reef chains can be considered; For the multitude of periodic energy passing over a reef chain, all products of the chain are compared, up to the length of the frame of the train.

\section{Wave Generation}
For Dune/Crest or Swing (stationary wave).\\
Supraliquid\\
Sail Effect / Surface Tension

\section{Phase Transitions}
For Dune/Spiral.
\begin{align}
z =  \sup_{U}\lvert curvature([a,b]) \rvert*(\cos{(isolation([a,b]))} +\mathrm{i} \sin{(visibility([a,b]))})
\end{align}

\section{Lightning}
For Spike/ZigZag.\\
Superconductor\\
If we're setting $\sqrt{1-y^2}$ to be the linear part of complex plane, $\phi$ in $z=\sqrt{1-y^2}e^{i\phi}$ is the rotational part.
\begin{align}
z =  straightness([a,b]) + \mathrm{i} curvature([a,b])
\end{align}

\section{Stateless}
Supercritical\\
gaseous/liquid carbondioxide

\iffalse
\printbibliography
\fi
\bibliography{Stopeight}{}
\bibliographystyle{plain}

\end{document}
